\documentclass[a4paper,12pt]{article}

\usepackage[french]{babel}   % Active la langue française
\usepackage{graphicx}    	% Pour insérer des images
\usepackage{lipsum}      	% Pour du texte factice
\usepackage{fancyhdr}    	% Pour personnaliser les en-têtes et pieds de page
\usepackage{listings}    	% Pour afficher le code
\usepackage{xcolor}      	% Pour personnaliser les couleurs
\usepackage{minted}      	% Pour une coloration syntaxique avancée
\usepackage{amsmath}     	% Pour les équations
\usepackage{hyperref}    	% Pour les liens cliquables
\usepackage{tcolorbox}
\usepackage{lmodern}


\hypersetup{
	colorlinks=true,   	% Active les liens en couleur
	linkcolor=blue,    	% Couleur des liens internes
	urlcolor=blue,     	% Couleur des liens externes
	citecolor=blue     	% Couleur des liens de citation
}



\pagestyle{fancy}  % Active le style personnalisé
\fancyhf{}  % Efface les en-têtes et pieds de page par défaut


\fancyfoot[R]{\includegraphics[width=1.3cm]{logo.png}}


% Ajouter la numérotation des pages à droite du pied de page
\fancyfoot[C]{\thepage}

\renewcommand{\sectionmark}[1]{\markboth{#1}{}}

% Afficher la section courante à gauche du pied de page
\fancyfoot[L]{\nouppercase{\leftmark}}

\begin{document}

% ---- Page de titre ----
\begin{titlepage}
	\centering
	\includegraphics[width=4cm]{logo.png} \\[1cm]  % Ajuste la taille du logo

	{\LARGE \textbf{Etude de la dynamique des foules}} \\[0.5cm]
    
	{\large Par le biais de la physique moderne} \\[1.5cm]

	\textbf{Auteurs :} \\[0.3cm]
	NADAUD Antonin \\
	JANINI  Raphaël \\
	LUBIN Thomas \\
	NUCE LAMOTHE Augustin \\[1cm]

	\textbf{Encadrant :} \\[0.3cm]
	DESPLAT Lucie \\[1.5cm]

	\textbf{\today}  % Affiche la date du jour
	\\[2cm]

	\vfill % Remplit l'espace verticalement pour centrer

	{\large CY Tech – préing-2 groupe 2} \\

\end{titlepage}

% --- page blanche ---
\newpage
\thispagestyle{empty}  % Pas de numérotation, ni d'en-tête/pied de page
\mbox{}  % Crée une page vide (avec un espace vide)

% ---- Sommaire ----
\newpage


\renewcommand{\contentsname}{Sommaire}  % Change "Contents" en "Sommaire"
\tableofcontents  % Génère le sommaire automatiquement avec des liens cliquables

% ---- introduction --- 
\newpage


\section{Situation}

\subsection{Methode de resolution equation}

 % \begin{tcolorbox}[colback=gray!20, colframe=black, colframe=white,sharp corners]
 
 % \end{tcolorbox}
 
\hspace{2em}

\indent Methode utiliser Ruge-Kuta à l'ordre 4

\hspace{2em} 

\begin{minted}[frame=single, bgcolor=white, linenos]{python}
"""
param:
	- nb_pt (entier) nombre de point (100 par défaut)
	- end (float) fin de l'interval sur [a,b] on prend b
	- start (float) debut de l'interval

return:
	tuple de list en premier l'axe x et en 2eme l'axe y	
"""
def solve_ordre4(nb_pt = 100, end=10, start=0):

    h = (end - start) / nb_pt
    x = [0]
    vitesse = [0]
    g = 9.81 

    for tn in range(nb_pt):

        f = lambda v : g - v

        k1 = h * f(vitesse[tn])
        k2 = h * f(vitesse[tn] +  .5 * k1)
        k3 = h * f(vitesse[tn] +  .5 * k2)
        k4 = h * f(vitesse[tn] + k3)

        vitesse.append(vitesse[tn] + 1/6 * ( k1 + 2 *k2 + 
        2 * k3 + k4))
        
        x.append(tn * h)
        
        return (x,vitesse)
\end{minted}





\subsection{Détails techniques}
\lipsum[3]






\section{Explication code}
\lipsum[4]

Référence à l'équation d'Einstein: \eqref{eq:einstein}


\section{Résultats}

\lipsum[5]



\section{Interprétation}
\lipsum[5]


\section{Analyse}

\lipsum[5]



\section{Bibliographie}

Voici la liste:

\vspace{1em}

\begin{itemize}
	\item \href{https://www.example.com}{Exemple 1}
	\item \href{https://www.google.com}{Google}
	\item \href{https://www.wikipedia.org}{Wikipedia}
\end{itemize}

\end{document}

\end{document}


pdflatex -shell-escape index.tex



