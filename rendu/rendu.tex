\documentclass[a4paper,12pt]{article}

\usepackage[french]{babel}   % Active la langue française
\usepackage{graphicx}    	% Pour insérer des images
\usepackage{lipsum}      	% Pour du texte factice
\usepackage{fancyhdr}    	% Pour personnaliser les en-têtes et pieds de page
\usepackage{listings}    	% Pour afficher le code
\usepackage{xcolor}      	% Pour personnaliser les couleurs
\usepackage{minted}      	% Pour une coloration syntaxique avancée
\usepackage{amsmath}     	% Pour les équations
\usepackage{hyperref}    	% Pour les liens cliquables
\usepackage{tcolorbox}
\usepackage{lmodern}


\hypersetup{
	colorlinks=true,   	% Active les liens en couleur
	linkcolor=blue,    	% Couleur des liens internes
	urlcolor=blue,     	% Couleur des liens externes
	citecolor=blue     	% Couleur des liens de citation
}



\pagestyle{fancy}  % Active le style personnalisé
\fancyhf{}  % Efface les en-têtes et pieds de page par défaut


\fancyfoot[R]{\includegraphics[width=1.3cm]{logo.png}}


% Ajouter la numérotation des pages à droite du pied de page
\fancyfoot[C]{\thepage}

\renewcommand{\sectionmark}[1]{\markboth{#1}{}}

% Afficher la section courante à gauche du pied de page
\fancyfoot[L]{\nouppercase{\leftmark}}

\begin{document}

% ---- Page de titre ----
\begin{titlepage}
	\centering
	\includegraphics[width=4cm]{logo.png} \\[1cm]  % Ajuste la taille du logo

	{\LARGE \textbf{Etude de la dynamique des foules}} \\[0.5cm]
    
	{\large Par le biais de la physique moderne} \\[1.5cm]

	\textbf{Auteurs :} \\[0.3cm]
	NADAUD Antonin \\
	JANINI  Raphaël \\
	LUBIN Thomas \\
	NUCE LAMOTHE Augustin \\[1cm]

	\textbf{Encadrant :} \\[0.3cm]
	DESPLAT Lucie \\[1.5cm]

	\textbf{\today}  % Affiche la date du jour
	\\[2cm]

	\vfill % Remplit l'espace verticalement pour centrer

	{\large CY Tech – préing-2 groupe 2} \\

\end{titlepage}

% --- page blanche ---
\newpage
\thispagestyle{empty}  % Pas de numérotation, ni d'en-tête/pied de page
\mbox{}  % Crée une page vide (avec un espace vide)

% ---- Sommaire ----
\newpage


\renewcommand{\contentsname}{Sommaire}  % Change "Contents" en "Sommaire"
\thispagestyle{empty}
\tableofcontents  % Génère le sommaire automatiquement avec des liens cliquables

% ---- introduction --- 
\newpage

\setcounter{page}{1} % -- met la page à 1 --

\section{Objectif}
\subsection{Situation 1}
\indent La première situation (imposée) dans le cadre de cette étude, consiste à modéliser l’évacuation d’une foule depuis un espace rectangulaire, tel qu’une salle. Les individus se dirigent vers une sortie sous l'effet d'une force motrice, tout en interagissant les uns avec les autres par le biais de forces sociales. Chaque individu est associé à une vitesse cible, représentant son intention de déplacement. L'objectif est d'avoir le temps d'évacuation totale pour la foulle

\subsection{Situation 2}
\indent On repart sur les bases de la situation une, mais cette fois ci on ajoute un obstacle à la sortie qui pourrait s'apparenter à un bureau. On ajoute aussi le sentiment de panique pour certains individus, modélisée par une vitesse plus important. L'objectif est de savoir combien de personne doivent paniquer en pourcentage pour avoir une évacuation optimale. On suppose que l'expérience se passe dans une entreprise

\
\section{Théorie sous-jacente}

Le modèle de force sociale, introduit par Helbing et Molnár\footnote{\url{https://doi.org/10.1103/PhysRevE.51.4282}}, vise à simuler le comportement de piétons dans des environnements denses, en représentant chaque individu comme une particule soumise à des forces comportementales. Deux forces fondamentales structurent ce modèle : la force motrice et la force sociale.

\paragraph{Force motrice.}
Chaque individu $i$ cherche à atteindre une vitesse souhaitée $\vec{v}_i^0$ dans une direction donnée. Ce comportement est modélisé par une force motrice, qui pousse l'individu à ajuster sa vitesse actuelle $\vec{v}_i$ à sa vitesse désirée, selon :

\begin{equation}
\label{eq:force_motrice}
\vec{f}_i^{\text{m}} = m_i \frac{\vec{v}_i^0 - \vec{v}_i}{\tau_i}
\end{equation}

où $m_i$ est la masse de l'individu et $\tau_i$ est un temps de relaxation caractérisant la rapidité avec laquelle l'individu tente d’atteindre sa vitesse cible.

\paragraph{Force sociale.}
Outre la volonté individuelle de se diriger vers une destination, les piétons interagissent entre eux via des forces dites sociales, traduisant leur tendance à maintenir une distance interpersonnelle. Ces interactions sont modélisées par une force répulsive de la forme :

\begin{equation}
\label{eq:force_sociale}
\vec{f}_{ij}^{\text{s}} = A_i \exp\left(\frac{r_{ij} - d_{ij}}{B_i}\right) \vec{n}_{ij}
\end{equation}

où :
\begin{itemize}
  \item $A_i$ et $B_i$ sont des constantes positives liées à l'intensité et à la portée de la répulsion,
  \item $r_{ij}$ est la somme des rayons corporels des individus $i$ et $j$,
  \item $d_{ij}$ est la distance entre les centres de masse des deux individus,
  \item $\vec{n}_{ij}$ est le vecteur unitaire pointant de $j$ vers $i$.
\end{itemize}

Ces forces permettent de simuler des comportements réalistes d’évitement et de gestion de l’espace dans des environnements contraints, comme lors d’une évacuation.


\section{Méthode}

\subsection{Determiner l'équation}

\indent Afin d’établir l’équation, nous avons tout d’abord supposé que le référentiel d’étude est galiléen, pour ensuite appliquer la seconde loi de newton $\sum \vec{F}_{\text{particule}} = m_{\text{particule}} \vec{a}$.
\\On a:
\[
\vec{f}_{\text{direction}} + \vec{f}_{\text{social}} = m_{\text{particule}} \frac{d\vec{v}}{dt}.
\]
\\ Au final après quelques simplification on a:

\begin{equation}
\label{eq:eq_diff}
\frac{\vec{v}_i^0 - \vec{v}_i}{\tau_i} + \frac{1}{m_{\text{particule}}} \sum_{j \neq i} A_i \exp\left( \frac{r_{ij} - d_{ij}}{B_i} \right) \vec{n}_{ij} = \frac{d\vec{v}}{dt}
\end{equation}

\subsection{Resolution de l'équation différentiel}

\indent Pour résoudre \eqref{eq:eq_diff}. Nous avons décider d'utiliser méthode d'euler:
\[
\vec{v}_{n+1} = \vec{v}_n + \Delta t \cdot \vec{F}_n
\]
\\ On introduit une personne modélisée en Python par un dictionnaire (pour mieux comprendre  la fonction qui permet de résoudre cette équation) :

\begin{minted}[frame=single, bgcolor=white, linenos]{python}
{
    "position": np.array([100 + 30 * y, 100 + 30 * x]),
	"masse": 10,
	"vitesse_desiree": 1.34, 
	"vitesse": np.array([0, 0]), #vitesse initiale
	"to": .2,
	"rayon": 10 + random.randint(-2, 2)
}
\end{minted}

\begin{itemize}
    \item \texttt{"position"} : Un tableau NumPy qui contient les coordonnées \([x, y]\) de la personne dans l'espace. 
    
    \item \texttt{"masse"} : La masse de la personne, ici fixée à 10 (arbitrairement).
    
    \item \texttt{"vitesse\_desiree"} : La vitesse désirée que la personne souhaite atteindre, (ici 1.34 $ms^{-1}$).
    
    \item \texttt{"vitesse"} : Un tableau NumPy qui représente la vitesse initiale de la personne. La vitesse initiale est définie comme un vecteur nul \([0, 0]\) (immobile au départ)
    
    \item \texttt{"to"} : Un paramètre fixé à 0.2. Il pourrait représenter un coefficient de friction, de résistance ou tout autre facteur de modification des interactions.
    
    \item \texttt{"rayon"} : Le rayon de la personne (representée comme un cercle), calculé comme 10 plus un nombre aléatoire compris entre -2 et 2. 
\end{itemize}

\newpage

\textbf{Le code qui permet de resoudre donc l'equation differentiel:}

\begin{minted}[frame=single, bgcolor=white, linenos]{python}
{
def euler(tab_personne, personne,indice, step=.02):

    
    f_totale = force_motrice(personne) #cacul de la force motrice
    f_totale += 1 /personne["masse"] * force_intercation_social_mur(personne , indice) 

    #projection sur Ux et Uy
    vitesse_x =  personne["vitesse"][0] + step * f_totale[0]
    vitesse_y = personne["vitesse"][1] + step * f_totale[1]
    
    #on actualise la position
    personne["position"] = np.array( [
        personne["position"][0] + vitesse_x,
        personne["position"][1] + vitesse_y 
    ])

    # v(t_n+1)
    personne["vitesse"] = np.array([
        vitesse_x,
        vitesse_y
    ])
}
\end{minted}

\noindent Dans les premières lignes on calcul la force totale.

\noindent Puis on applique la méthode d'euler:
\begin{equation}
    \vec{v}_{n+1} = \vec{v}_n + \Delta t \cdot \vec{F_{totale}}
\end{equation}

\noindent Pour chaque direction $(0_y)$ et $(0_x)$ on projette pour avoir la vitesse en composante respective x et y. (ligne 9 et 10), ici $\Delta t = 0.02$ ce qui minimise la marge d'erreur.

\

\noindent La dernière ligne permet d'actualiser 

\newpage

\section{Bibliographie}

Voici la liste:

\vspace{1em}

\begin{itemize}
	\item \href{https://journals.aps.org/pre/abstract/10.1103/PhysRevE.51.4282}{Social force model for pedestrian dynamics (Dirk Helbing et Péter Molnár)}
	\item \href{https://www.google.com}{Google}
	\item \href{https://www.wikipedia.org}{Wikipedia}
\end{itemize}

\end{document}

\end{document}


pdflatex -shell-escape index.tex



